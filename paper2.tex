% Options for packages loaded elsewhere
\PassOptionsToPackage{unicode}{hyperref}
\PassOptionsToPackage{hyphens}{url}
\PassOptionsToPackage{dvipsnames,svgnames,x11names}{xcolor}
%
\documentclass[
  letterpaper,
  DIV=11,
  numbers=noendperiod]{scrartcl}

\usepackage{amsmath,amssymb}
\usepackage{iftex}
\ifPDFTeX
  \usepackage[T1]{fontenc}
  \usepackage[utf8]{inputenc}
  \usepackage{textcomp} % provide euro and other symbols
\else % if luatex or xetex
  \usepackage{unicode-math}
  \defaultfontfeatures{Scale=MatchLowercase}
  \defaultfontfeatures[\rmfamily]{Ligatures=TeX,Scale=1}
\fi
\usepackage{lmodern}
\ifPDFTeX\else  
    % xetex/luatex font selection
\fi
% Use upquote if available, for straight quotes in verbatim environments
\IfFileExists{upquote.sty}{\usepackage{upquote}}{}
\IfFileExists{microtype.sty}{% use microtype if available
  \usepackage[]{microtype}
  \UseMicrotypeSet[protrusion]{basicmath} % disable protrusion for tt fonts
}{}
\makeatletter
\@ifundefined{KOMAClassName}{% if non-KOMA class
  \IfFileExists{parskip.sty}{%
    \usepackage{parskip}
  }{% else
    \setlength{\parindent}{0pt}
    \setlength{\parskip}{6pt plus 2pt minus 1pt}}
}{% if KOMA class
  \KOMAoptions{parskip=half}}
\makeatother
\usepackage{xcolor}
\setlength{\emergencystretch}{3em} % prevent overfull lines
\setcounter{secnumdepth}{5}
% Make \paragraph and \subparagraph free-standing
\ifx\paragraph\undefined\else
  \let\oldparagraph\paragraph
  \renewcommand{\paragraph}[1]{\oldparagraph{#1}\mbox{}}
\fi
\ifx\subparagraph\undefined\else
  \let\oldsubparagraph\subparagraph
  \renewcommand{\subparagraph}[1]{\oldsubparagraph{#1}\mbox{}}
\fi


\providecommand{\tightlist}{%
  \setlength{\itemsep}{0pt}\setlength{\parskip}{0pt}}\usepackage{longtable,booktabs,array}
\usepackage{calc} % for calculating minipage widths
% Correct order of tables after \paragraph or \subparagraph
\usepackage{etoolbox}
\makeatletter
\patchcmd\longtable{\par}{\if@noskipsec\mbox{}\fi\par}{}{}
\makeatother
% Allow footnotes in longtable head/foot
\IfFileExists{footnotehyper.sty}{\usepackage{footnotehyper}}{\usepackage{footnote}}
\makesavenoteenv{longtable}
\usepackage{graphicx}
\makeatletter
\def\maxwidth{\ifdim\Gin@nat@width>\linewidth\linewidth\else\Gin@nat@width\fi}
\def\maxheight{\ifdim\Gin@nat@height>\textheight\textheight\else\Gin@nat@height\fi}
\makeatother
% Scale images if necessary, so that they will not overflow the page
% margins by default, and it is still possible to overwrite the defaults
% using explicit options in \includegraphics[width, height, ...]{}
\setkeys{Gin}{width=\maxwidth,height=\maxheight,keepaspectratio}
% Set default figure placement to htbp
\makeatletter
\def\fps@figure{htbp}
\makeatother
% definitions for citeproc citations
\NewDocumentCommand\citeproctext{}{}
\NewDocumentCommand\citeproc{mm}{%
  \begingroup\def\citeproctext{#2}\cite{#1}\endgroup}
\makeatletter
 % allow citations to break across lines
 \let\@cite@ofmt\@firstofone
 % avoid brackets around text for \cite:
 \def\@biblabel#1{}
 \def\@cite#1#2{{#1\if@tempswa , #2\fi}}
\makeatother
\newlength{\cslhangindent}
\setlength{\cslhangindent}{1.5em}
\newlength{\csllabelwidth}
\setlength{\csllabelwidth}{3em}
\newenvironment{CSLReferences}[2] % #1 hanging-indent, #2 entry-spacing
 {\begin{list}{}{%
  \setlength{\itemindent}{0pt}
  \setlength{\leftmargin}{0pt}
  \setlength{\parsep}{0pt}
  % turn on hanging indent if param 1 is 1
  \ifodd #1
   \setlength{\leftmargin}{\cslhangindent}
   \setlength{\itemindent}{-1\cslhangindent}
  \fi
  % set entry spacing
  \setlength{\itemsep}{#2\baselineskip}}}
 {\end{list}}
\usepackage{calc}
\newcommand{\CSLBlock}[1]{\hfill\break\parbox[t]{\linewidth}{\strut\ignorespaces#1\strut}}
\newcommand{\CSLLeftMargin}[1]{\parbox[t]{\csllabelwidth}{\strut#1\strut}}
\newcommand{\CSLRightInline}[1]{\parbox[t]{\linewidth - \csllabelwidth}{\strut#1\strut}}
\newcommand{\CSLIndent}[1]{\hspace{\cslhangindent}#1}

\usepackage{float}
\usepackage{tabularray}
\usepackage[normalem]{ulem}
\usepackage{graphicx}
\UseTblrLibrary{booktabs}
\UseTblrLibrary{siunitx}
\NewTableCommand{\tinytableDefineColor}[3]{\definecolor{#1}{#2}{#3}}
\newcommand{\tinytableTabularrayUnderline}[1]{\underline{#1}}
\newcommand{\tinytableTabularrayStrikeout}[1]{\sout{#1}}
\KOMAoption{captions}{tableheading}
\makeatletter
\@ifpackageloaded{caption}{}{\usepackage{caption}}
\AtBeginDocument{%
\ifdefined\contentsname
  \renewcommand*\contentsname{Table of contents}
\else
  \newcommand\contentsname{Table of contents}
\fi
\ifdefined\listfigurename
  \renewcommand*\listfigurename{List of Figures}
\else
  \newcommand\listfigurename{List of Figures}
\fi
\ifdefined\listtablename
  \renewcommand*\listtablename{List of Tables}
\else
  \newcommand\listtablename{List of Tables}
\fi
\ifdefined\figurename
  \renewcommand*\figurename{Figure}
\else
  \newcommand\figurename{Figure}
\fi
\ifdefined\tablename
  \renewcommand*\tablename{Table}
\else
  \newcommand\tablename{Table}
\fi
}
\@ifpackageloaded{float}{}{\usepackage{float}}
\floatstyle{ruled}
\@ifundefined{c@chapter}{\newfloat{codelisting}{h}{lop}}{\newfloat{codelisting}{h}{lop}[chapter]}
\floatname{codelisting}{Listing}
\newcommand*\listoflistings{\listof{codelisting}{List of Listings}}
\makeatother
\makeatletter
\makeatother
\makeatletter
\@ifpackageloaded{caption}{}{\usepackage{caption}}
\@ifpackageloaded{subcaption}{}{\usepackage{subcaption}}
\makeatother
\ifLuaTeX
  \usepackage{selnolig}  % disable illegal ligatures
\fi
\usepackage{bookmark}

\IfFileExists{xurl.sty}{\usepackage{xurl}}{} % add URL line breaks if available
\urlstyle{same} % disable monospaced font for URLs
\hypersetup{
  pdftitle={Predition on total expected points added on pass attempts and sacks},
  pdfauthor={Yunshu Zhang},
  colorlinks=true,
  linkcolor={blue},
  filecolor={Maroon},
  citecolor={Blue},
  urlcolor={Blue},
  pdfcreator={LaTeX via pandoc}}

\title{Predition on total expected points added on pass attempts and
sacks\thanks{Code and data are available at:
https://github.com/Yunshu921/predicction\_passing\_epa.git.}}
\author{Yunshu Zhang}
\date{April 4, 2024}

\begin{document}
\maketitle

\renewcommand*\contentsname{Table of contents}
{
\hypersetup{linkcolor=}
\setcounter{tocdepth}{3}
\tableofcontents
}
\section{Introduction}\label{introduction}

Rugby is a popular worldwide sport that has a sizable following across
many different countries and regions. What's more, rugby is one of the
most famous sports in the world because of its elite professional
league, thrilling international competitions, and inclusion on the
Olympic Games schedule. One thing need to mention is that rugby's growth
and popularity have enhanced not only the sporting culture but also
worldwide cooperation and exchanges while igniting people's passion and
providing limitless enjoyment.

By Niko's (Besnier 2014) article, we realized that in various regions of
the world, sports migration has gained significant importance on a
number of levels. Even though there is very little chance of success in
the sports industry, children and young people in many Global South
societies are drawn to success stories in the media and increasingly
define their future in terms of the possibility of becoming professional
rugby players, baseball players, or transnational football players. For
the United States, football has always been a high-profile sport, and in
this historical context we focus our attention on the use of
mathematical statistical tools to predict sporting events. Specifically,
we use \texttt{nflreadr} (n.d.) to predict the total score of the NFL in
the 2023 WEEK 14.

\section{Model}\label{model}

From Table~\ref{tbl-model-1}, we can write down our model equation as
following. \[
Y = β_{0} + β_{1} \times X_{1} + β_{2} \times X_{2} + \epsilon
\] where:

\begin{itemize}
\tightlist
\item
  \(Y\) is the total expected points added on pass attempts and sacks.
\item
  \(β_{0}\) is the intercept which means the total expected points
  without any rush attempts and yards which is -1.447.
\item
  \(β_{1}\) is the amount of point changed for one rush attempt which is
  0.695.
\item
  \(β_{2}\) is the amount of point changed for one yard which is -0.096.
\item
  \(X_{1}\) is the number of official rush attempts (incl.~scrambles and
  kneel downs). Rushes after a lateral reception don't count as carry.
\item
  \(X_{2}\) are yards gained when rushing with the ball (incl.~scrambles
  and kneel downs). Also includes yards gained after obtaining a lateral
  on a play that started with a rushing attempt.
\item
  \(\epsilon\) represents the error term
\end{itemize}

\begin{table}

\caption{\label{tbl-model-1}Modeling the expected point for Week 14 of
NFL, 2023}

\centering{

\centering
\begin{tblr}[         %% tabularray outer open
]                     %% tabularray outer close
{                     %% tabularray inner open
colspec={Q[]Q[]},
column{1}={halign=l,},
column{2}={halign=c,},
hline{8}={1,2}{solid, 0.05em, black},
}                     %% tabularray inner close
\toprule
& Model \\ \midrule %% TinyTableHeader
(Intercept)     & \num{-1.447}    \\
& (\num{0.775})   \\
carries         & \num{0.695}     \\
& (\num{0.256})   \\
rushing\_yards & \num{-0.096}    \\
& (\num{0.043})   \\
Num.Obs.        & \num{393}       \\
R2              & \num{0.019}     \\
R2 Adj.         & \num{0.013}     \\
AIC             & \num{2919.8}    \\
BIC             & \num{2935.7}    \\
Log.Lik.        & \num{-1455.907} \\
RMSE            & \num{9.83}      \\
\bottomrule
\end{tblr}

}

\end{table}%

\phantomsection\label{refs}
\begin{CSLReferences}{1}{0}
\bibitem[\citeproctext]{ref-nfl}
n.d. \emph{Nflreadr.nflverse.com}.
\url{https://nflreadr.nflverse.com/articles/dictionary_player_stats.html}.

\bibitem[\citeproctext]{ref-rugby}
Besnier, Niko. 2014. {``Pacific Island Rugby: Histories, Mobilities,
Comparisons.''} \emph{Asia Pacific Journal of Sport and Social Science}
3 (3): 268--76.
https://doi.org/\url{https://doi.org/10.1080/21640599.2014.982894}.

\end{CSLReferences}



\end{document}
